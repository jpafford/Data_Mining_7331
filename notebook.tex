
% Default to the notebook output style

    


% Inherit from the specified cell style.




    
\documentclass[11pt]{article}

    
    
    \usepackage[T1]{fontenc}
    % Nicer default font (+ math font) than Computer Modern for most use cases
    \usepackage{mathpazo}

    % Basic figure setup, for now with no caption control since it's done
    % automatically by Pandoc (which extracts ![](path) syntax from Markdown).
    \usepackage{graphicx}
    % We will generate all images so they have a width \maxwidth. This means
    % that they will get their normal width if they fit onto the page, but
    % are scaled down if they would overflow the margins.
    \makeatletter
    \def\maxwidth{\ifdim\Gin@nat@width>\linewidth\linewidth
    \else\Gin@nat@width\fi}
    \makeatother
    \let\Oldincludegraphics\includegraphics
    % Set max figure width to be 80% of text width, for now hardcoded.
    \renewcommand{\includegraphics}[1]{\Oldincludegraphics[width=.8\maxwidth]{#1}}
    % Ensure that by default, figures have no caption (until we provide a
    % proper Figure object with a Caption API and a way to capture that
    % in the conversion process - todo).
    \usepackage{caption}
    \DeclareCaptionLabelFormat{nolabel}{}
    \captionsetup{labelformat=nolabel}

    \usepackage{adjustbox} % Used to constrain images to a maximum size 
    \usepackage{xcolor} % Allow colors to be defined
    \usepackage{enumerate} % Needed for markdown enumerations to work
    \usepackage{geometry} % Used to adjust the document margins
    \usepackage{amsmath} % Equations
    \usepackage{amssymb} % Equations
    \usepackage{textcomp} % defines textquotesingle
    % Hack from http://tex.stackexchange.com/a/47451/13684:
    \AtBeginDocument{%
        \def\PYZsq{\textquotesingle}% Upright quotes in Pygmentized code
    }
    \usepackage{upquote} % Upright quotes for verbatim code
    \usepackage{eurosym} % defines \euro
    \usepackage[mathletters]{ucs} % Extended unicode (utf-8) support
    \usepackage[utf8x]{inputenc} % Allow utf-8 characters in the tex document
    \usepackage{fancyvrb} % verbatim replacement that allows latex
    \usepackage{grffile} % extends the file name processing of package graphics 
                         % to support a larger range 
    % The hyperref package gives us a pdf with properly built
    % internal navigation ('pdf bookmarks' for the table of contents,
    % internal cross-reference links, web links for URLs, etc.)
    \usepackage{hyperref}
    \usepackage{longtable} % longtable support required by pandoc >1.10
    \usepackage{booktabs}  % table support for pandoc > 1.12.2
    \usepackage[inline]{enumitem} % IRkernel/repr support (it uses the enumerate* environment)
    \usepackage[normalem]{ulem} % ulem is needed to support strikethroughs (\sout)
                                % normalem makes italics be italics, not underlines
    

    
    
    % Colors for the hyperref package
    \definecolor{urlcolor}{rgb}{0,.145,.698}
    \definecolor{linkcolor}{rgb}{.71,0.21,0.01}
    \definecolor{citecolor}{rgb}{.12,.54,.11}

    % ANSI colors
    \definecolor{ansi-black}{HTML}{3E424D}
    \definecolor{ansi-black-intense}{HTML}{282C36}
    \definecolor{ansi-red}{HTML}{E75C58}
    \definecolor{ansi-red-intense}{HTML}{B22B31}
    \definecolor{ansi-green}{HTML}{00A250}
    \definecolor{ansi-green-intense}{HTML}{007427}
    \definecolor{ansi-yellow}{HTML}{DDB62B}
    \definecolor{ansi-yellow-intense}{HTML}{B27D12}
    \definecolor{ansi-blue}{HTML}{208FFB}
    \definecolor{ansi-blue-intense}{HTML}{0065CA}
    \definecolor{ansi-magenta}{HTML}{D160C4}
    \definecolor{ansi-magenta-intense}{HTML}{A03196}
    \definecolor{ansi-cyan}{HTML}{60C6C8}
    \definecolor{ansi-cyan-intense}{HTML}{258F8F}
    \definecolor{ansi-white}{HTML}{C5C1B4}
    \definecolor{ansi-white-intense}{HTML}{A1A6B2}

    % commands and environments needed by pandoc snippets
    % extracted from the output of `pandoc -s`
    \providecommand{\tightlist}{%
      \setlength{\itemsep}{0pt}\setlength{\parskip}{0pt}}
    \DefineVerbatimEnvironment{Highlighting}{Verbatim}{commandchars=\\\{\}}
    % Add ',fontsize=\small' for more characters per line
    \newenvironment{Shaded}{}{}
    \newcommand{\KeywordTok}[1]{\textcolor[rgb]{0.00,0.44,0.13}{\textbf{{#1}}}}
    \newcommand{\DataTypeTok}[1]{\textcolor[rgb]{0.56,0.13,0.00}{{#1}}}
    \newcommand{\DecValTok}[1]{\textcolor[rgb]{0.25,0.63,0.44}{{#1}}}
    \newcommand{\BaseNTok}[1]{\textcolor[rgb]{0.25,0.63,0.44}{{#1}}}
    \newcommand{\FloatTok}[1]{\textcolor[rgb]{0.25,0.63,0.44}{{#1}}}
    \newcommand{\CharTok}[1]{\textcolor[rgb]{0.25,0.44,0.63}{{#1}}}
    \newcommand{\StringTok}[1]{\textcolor[rgb]{0.25,0.44,0.63}{{#1}}}
    \newcommand{\CommentTok}[1]{\textcolor[rgb]{0.38,0.63,0.69}{\textit{{#1}}}}
    \newcommand{\OtherTok}[1]{\textcolor[rgb]{0.00,0.44,0.13}{{#1}}}
    \newcommand{\AlertTok}[1]{\textcolor[rgb]{1.00,0.00,0.00}{\textbf{{#1}}}}
    \newcommand{\FunctionTok}[1]{\textcolor[rgb]{0.02,0.16,0.49}{{#1}}}
    \newcommand{\RegionMarkerTok}[1]{{#1}}
    \newcommand{\ErrorTok}[1]{\textcolor[rgb]{1.00,0.00,0.00}{\textbf{{#1}}}}
    \newcommand{\NormalTok}[1]{{#1}}
    
    % Additional commands for more recent versions of Pandoc
    \newcommand{\ConstantTok}[1]{\textcolor[rgb]{0.53,0.00,0.00}{{#1}}}
    \newcommand{\SpecialCharTok}[1]{\textcolor[rgb]{0.25,0.44,0.63}{{#1}}}
    \newcommand{\VerbatimStringTok}[1]{\textcolor[rgb]{0.25,0.44,0.63}{{#1}}}
    \newcommand{\SpecialStringTok}[1]{\textcolor[rgb]{0.73,0.40,0.53}{{#1}}}
    \newcommand{\ImportTok}[1]{{#1}}
    \newcommand{\DocumentationTok}[1]{\textcolor[rgb]{0.73,0.13,0.13}{\textit{{#1}}}}
    \newcommand{\AnnotationTok}[1]{\textcolor[rgb]{0.38,0.63,0.69}{\textbf{\textit{{#1}}}}}
    \newcommand{\CommentVarTok}[1]{\textcolor[rgb]{0.38,0.63,0.69}{\textbf{\textit{{#1}}}}}
    \newcommand{\VariableTok}[1]{\textcolor[rgb]{0.10,0.09,0.49}{{#1}}}
    \newcommand{\ControlFlowTok}[1]{\textcolor[rgb]{0.00,0.44,0.13}{\textbf{{#1}}}}
    \newcommand{\OperatorTok}[1]{\textcolor[rgb]{0.40,0.40,0.40}{{#1}}}
    \newcommand{\BuiltInTok}[1]{{#1}}
    \newcommand{\ExtensionTok}[1]{{#1}}
    \newcommand{\PreprocessorTok}[1]{\textcolor[rgb]{0.74,0.48,0.00}{{#1}}}
    \newcommand{\AttributeTok}[1]{\textcolor[rgb]{0.49,0.56,0.16}{{#1}}}
    \newcommand{\InformationTok}[1]{\textcolor[rgb]{0.38,0.63,0.69}{\textbf{\textit{{#1}}}}}
    \newcommand{\WarningTok}[1]{\textcolor[rgb]{0.38,0.63,0.69}{\textbf{\textit{{#1}}}}}
    
    
    % Define a nice break command that doesn't care if a line doesn't already
    % exist.
    \def\br{\hspace*{\fill} \\* }
    % Math Jax compatability definitions
    \def\gt{>}
    \def\lt{<}
    % Document parameters
    \title{Lab One Visualization and Data Preprocessing}
    
    
    

    % Pygments definitions
    
\makeatletter
\def\PY@reset{\let\PY@it=\relax \let\PY@bf=\relax%
    \let\PY@ul=\relax \let\PY@tc=\relax%
    \let\PY@bc=\relax \let\PY@ff=\relax}
\def\PY@tok#1{\csname PY@tok@#1\endcsname}
\def\PY@toks#1+{\ifx\relax#1\empty\else%
    \PY@tok{#1}\expandafter\PY@toks\fi}
\def\PY@do#1{\PY@bc{\PY@tc{\PY@ul{%
    \PY@it{\PY@bf{\PY@ff{#1}}}}}}}
\def\PY#1#2{\PY@reset\PY@toks#1+\relax+\PY@do{#2}}

\expandafter\def\csname PY@tok@w\endcsname{\def\PY@tc##1{\textcolor[rgb]{0.73,0.73,0.73}{##1}}}
\expandafter\def\csname PY@tok@c\endcsname{\let\PY@it=\textit\def\PY@tc##1{\textcolor[rgb]{0.25,0.50,0.50}{##1}}}
\expandafter\def\csname PY@tok@cp\endcsname{\def\PY@tc##1{\textcolor[rgb]{0.74,0.48,0.00}{##1}}}
\expandafter\def\csname PY@tok@k\endcsname{\let\PY@bf=\textbf\def\PY@tc##1{\textcolor[rgb]{0.00,0.50,0.00}{##1}}}
\expandafter\def\csname PY@tok@kp\endcsname{\def\PY@tc##1{\textcolor[rgb]{0.00,0.50,0.00}{##1}}}
\expandafter\def\csname PY@tok@kt\endcsname{\def\PY@tc##1{\textcolor[rgb]{0.69,0.00,0.25}{##1}}}
\expandafter\def\csname PY@tok@o\endcsname{\def\PY@tc##1{\textcolor[rgb]{0.40,0.40,0.40}{##1}}}
\expandafter\def\csname PY@tok@ow\endcsname{\let\PY@bf=\textbf\def\PY@tc##1{\textcolor[rgb]{0.67,0.13,1.00}{##1}}}
\expandafter\def\csname PY@tok@nb\endcsname{\def\PY@tc##1{\textcolor[rgb]{0.00,0.50,0.00}{##1}}}
\expandafter\def\csname PY@tok@nf\endcsname{\def\PY@tc##1{\textcolor[rgb]{0.00,0.00,1.00}{##1}}}
\expandafter\def\csname PY@tok@nc\endcsname{\let\PY@bf=\textbf\def\PY@tc##1{\textcolor[rgb]{0.00,0.00,1.00}{##1}}}
\expandafter\def\csname PY@tok@nn\endcsname{\let\PY@bf=\textbf\def\PY@tc##1{\textcolor[rgb]{0.00,0.00,1.00}{##1}}}
\expandafter\def\csname PY@tok@ne\endcsname{\let\PY@bf=\textbf\def\PY@tc##1{\textcolor[rgb]{0.82,0.25,0.23}{##1}}}
\expandafter\def\csname PY@tok@nv\endcsname{\def\PY@tc##1{\textcolor[rgb]{0.10,0.09,0.49}{##1}}}
\expandafter\def\csname PY@tok@no\endcsname{\def\PY@tc##1{\textcolor[rgb]{0.53,0.00,0.00}{##1}}}
\expandafter\def\csname PY@tok@nl\endcsname{\def\PY@tc##1{\textcolor[rgb]{0.63,0.63,0.00}{##1}}}
\expandafter\def\csname PY@tok@ni\endcsname{\let\PY@bf=\textbf\def\PY@tc##1{\textcolor[rgb]{0.60,0.60,0.60}{##1}}}
\expandafter\def\csname PY@tok@na\endcsname{\def\PY@tc##1{\textcolor[rgb]{0.49,0.56,0.16}{##1}}}
\expandafter\def\csname PY@tok@nt\endcsname{\let\PY@bf=\textbf\def\PY@tc##1{\textcolor[rgb]{0.00,0.50,0.00}{##1}}}
\expandafter\def\csname PY@tok@nd\endcsname{\def\PY@tc##1{\textcolor[rgb]{0.67,0.13,1.00}{##1}}}
\expandafter\def\csname PY@tok@s\endcsname{\def\PY@tc##1{\textcolor[rgb]{0.73,0.13,0.13}{##1}}}
\expandafter\def\csname PY@tok@sd\endcsname{\let\PY@it=\textit\def\PY@tc##1{\textcolor[rgb]{0.73,0.13,0.13}{##1}}}
\expandafter\def\csname PY@tok@si\endcsname{\let\PY@bf=\textbf\def\PY@tc##1{\textcolor[rgb]{0.73,0.40,0.53}{##1}}}
\expandafter\def\csname PY@tok@se\endcsname{\let\PY@bf=\textbf\def\PY@tc##1{\textcolor[rgb]{0.73,0.40,0.13}{##1}}}
\expandafter\def\csname PY@tok@sr\endcsname{\def\PY@tc##1{\textcolor[rgb]{0.73,0.40,0.53}{##1}}}
\expandafter\def\csname PY@tok@ss\endcsname{\def\PY@tc##1{\textcolor[rgb]{0.10,0.09,0.49}{##1}}}
\expandafter\def\csname PY@tok@sx\endcsname{\def\PY@tc##1{\textcolor[rgb]{0.00,0.50,0.00}{##1}}}
\expandafter\def\csname PY@tok@m\endcsname{\def\PY@tc##1{\textcolor[rgb]{0.40,0.40,0.40}{##1}}}
\expandafter\def\csname PY@tok@gh\endcsname{\let\PY@bf=\textbf\def\PY@tc##1{\textcolor[rgb]{0.00,0.00,0.50}{##1}}}
\expandafter\def\csname PY@tok@gu\endcsname{\let\PY@bf=\textbf\def\PY@tc##1{\textcolor[rgb]{0.50,0.00,0.50}{##1}}}
\expandafter\def\csname PY@tok@gd\endcsname{\def\PY@tc##1{\textcolor[rgb]{0.63,0.00,0.00}{##1}}}
\expandafter\def\csname PY@tok@gi\endcsname{\def\PY@tc##1{\textcolor[rgb]{0.00,0.63,0.00}{##1}}}
\expandafter\def\csname PY@tok@gr\endcsname{\def\PY@tc##1{\textcolor[rgb]{1.00,0.00,0.00}{##1}}}
\expandafter\def\csname PY@tok@ge\endcsname{\let\PY@it=\textit}
\expandafter\def\csname PY@tok@gs\endcsname{\let\PY@bf=\textbf}
\expandafter\def\csname PY@tok@gp\endcsname{\let\PY@bf=\textbf\def\PY@tc##1{\textcolor[rgb]{0.00,0.00,0.50}{##1}}}
\expandafter\def\csname PY@tok@go\endcsname{\def\PY@tc##1{\textcolor[rgb]{0.53,0.53,0.53}{##1}}}
\expandafter\def\csname PY@tok@gt\endcsname{\def\PY@tc##1{\textcolor[rgb]{0.00,0.27,0.87}{##1}}}
\expandafter\def\csname PY@tok@err\endcsname{\def\PY@bc##1{\setlength{\fboxsep}{0pt}\fcolorbox[rgb]{1.00,0.00,0.00}{1,1,1}{\strut ##1}}}
\expandafter\def\csname PY@tok@kc\endcsname{\let\PY@bf=\textbf\def\PY@tc##1{\textcolor[rgb]{0.00,0.50,0.00}{##1}}}
\expandafter\def\csname PY@tok@kd\endcsname{\let\PY@bf=\textbf\def\PY@tc##1{\textcolor[rgb]{0.00,0.50,0.00}{##1}}}
\expandafter\def\csname PY@tok@kn\endcsname{\let\PY@bf=\textbf\def\PY@tc##1{\textcolor[rgb]{0.00,0.50,0.00}{##1}}}
\expandafter\def\csname PY@tok@kr\endcsname{\let\PY@bf=\textbf\def\PY@tc##1{\textcolor[rgb]{0.00,0.50,0.00}{##1}}}
\expandafter\def\csname PY@tok@bp\endcsname{\def\PY@tc##1{\textcolor[rgb]{0.00,0.50,0.00}{##1}}}
\expandafter\def\csname PY@tok@fm\endcsname{\def\PY@tc##1{\textcolor[rgb]{0.00,0.00,1.00}{##1}}}
\expandafter\def\csname PY@tok@vc\endcsname{\def\PY@tc##1{\textcolor[rgb]{0.10,0.09,0.49}{##1}}}
\expandafter\def\csname PY@tok@vg\endcsname{\def\PY@tc##1{\textcolor[rgb]{0.10,0.09,0.49}{##1}}}
\expandafter\def\csname PY@tok@vi\endcsname{\def\PY@tc##1{\textcolor[rgb]{0.10,0.09,0.49}{##1}}}
\expandafter\def\csname PY@tok@vm\endcsname{\def\PY@tc##1{\textcolor[rgb]{0.10,0.09,0.49}{##1}}}
\expandafter\def\csname PY@tok@sa\endcsname{\def\PY@tc##1{\textcolor[rgb]{0.73,0.13,0.13}{##1}}}
\expandafter\def\csname PY@tok@sb\endcsname{\def\PY@tc##1{\textcolor[rgb]{0.73,0.13,0.13}{##1}}}
\expandafter\def\csname PY@tok@sc\endcsname{\def\PY@tc##1{\textcolor[rgb]{0.73,0.13,0.13}{##1}}}
\expandafter\def\csname PY@tok@dl\endcsname{\def\PY@tc##1{\textcolor[rgb]{0.73,0.13,0.13}{##1}}}
\expandafter\def\csname PY@tok@s2\endcsname{\def\PY@tc##1{\textcolor[rgb]{0.73,0.13,0.13}{##1}}}
\expandafter\def\csname PY@tok@sh\endcsname{\def\PY@tc##1{\textcolor[rgb]{0.73,0.13,0.13}{##1}}}
\expandafter\def\csname PY@tok@s1\endcsname{\def\PY@tc##1{\textcolor[rgb]{0.73,0.13,0.13}{##1}}}
\expandafter\def\csname PY@tok@mb\endcsname{\def\PY@tc##1{\textcolor[rgb]{0.40,0.40,0.40}{##1}}}
\expandafter\def\csname PY@tok@mf\endcsname{\def\PY@tc##1{\textcolor[rgb]{0.40,0.40,0.40}{##1}}}
\expandafter\def\csname PY@tok@mh\endcsname{\def\PY@tc##1{\textcolor[rgb]{0.40,0.40,0.40}{##1}}}
\expandafter\def\csname PY@tok@mi\endcsname{\def\PY@tc##1{\textcolor[rgb]{0.40,0.40,0.40}{##1}}}
\expandafter\def\csname PY@tok@il\endcsname{\def\PY@tc##1{\textcolor[rgb]{0.40,0.40,0.40}{##1}}}
\expandafter\def\csname PY@tok@mo\endcsname{\def\PY@tc##1{\textcolor[rgb]{0.40,0.40,0.40}{##1}}}
\expandafter\def\csname PY@tok@ch\endcsname{\let\PY@it=\textit\def\PY@tc##1{\textcolor[rgb]{0.25,0.50,0.50}{##1}}}
\expandafter\def\csname PY@tok@cm\endcsname{\let\PY@it=\textit\def\PY@tc##1{\textcolor[rgb]{0.25,0.50,0.50}{##1}}}
\expandafter\def\csname PY@tok@cpf\endcsname{\let\PY@it=\textit\def\PY@tc##1{\textcolor[rgb]{0.25,0.50,0.50}{##1}}}
\expandafter\def\csname PY@tok@c1\endcsname{\let\PY@it=\textit\def\PY@tc##1{\textcolor[rgb]{0.25,0.50,0.50}{##1}}}
\expandafter\def\csname PY@tok@cs\endcsname{\let\PY@it=\textit\def\PY@tc##1{\textcolor[rgb]{0.25,0.50,0.50}{##1}}}

\def\PYZbs{\char`\\}
\def\PYZus{\char`\_}
\def\PYZob{\char`\{}
\def\PYZcb{\char`\}}
\def\PYZca{\char`\^}
\def\PYZam{\char`\&}
\def\PYZlt{\char`\<}
\def\PYZgt{\char`\>}
\def\PYZsh{\char`\#}
\def\PYZpc{\char`\%}
\def\PYZdl{\char`\$}
\def\PYZhy{\char`\-}
\def\PYZsq{\char`\'}
\def\PYZdq{\char`\"}
\def\PYZti{\char`\~}
% for compatibility with earlier versions
\def\PYZat{@}
\def\PYZlb{[}
\def\PYZrb{]}
\makeatother


    % Exact colors from NB
    \definecolor{incolor}{rgb}{0.0, 0.0, 0.5}
    \definecolor{outcolor}{rgb}{0.545, 0.0, 0.0}



    
    % Prevent overflowing lines due to hard-to-break entities
    \sloppy 
    % Setup hyperref package
    \hypersetup{
      breaklinks=true,  % so long urls are correctly broken across lines
      colorlinks=true,
      urlcolor=urlcolor,
      linkcolor=linkcolor,
      citecolor=citecolor,
      }
    % Slightly bigger margins than the latex defaults
    
    \geometry{verbose,tmargin=1in,bmargin=1in,lmargin=1in,rmargin=1in}
    
    

    \begin{document}
    
    
    \maketitle
    
    

    
    \section{Lab One Visualization and Data
Preprocessing}\label{lab-one-visualization-and-data-preprocessing}

\begin{itemize}
\tightlist
\item
  Ho, Andy
\item
  Nguyen, An
\item
  Pafford, Jodi
\item
  Wheelis, Tori
\end{itemize}

    \section{Business Understanding (10)}\label{business-understanding-10}

This dataset contains information from schools within the California
Department of Education. It specifically addresses school progress on
standardized tests in Mathematics during the 2017-2018 school year. In
part, the data will be used to support Local Education Agencies (LEA) in
identifying strengths, weaknesses, and areas for improvement; assist in
determining whether LEAs are eligible for assistance; and assist the
SSPI in determining whether LEAs are eligible for more intensive state
support/intervention. The school accountability system in California has
10 priorities. This dataset looks at one part of the 4th priority:
Student Achievement. The dataset used in this analysis pertains
specifically to its Mathematics curriculum and the progress made from
2017 school year and the 2018 school year.\\
From this dataset, we will look at how the state, each county, LEA, and
campus increased or decreased their average distance above or below the
passing standard score. This dataset also allows us to look at different
programs (English Learner, Socioeconomic Disadvantaged, Students with
Disabilities, Foster Youth, Homeless Youth) and race/ethnicities
(Black/African American, American Indian or Alaska Native, Asian,
Filipino, Hispanic, Pacific Islander, White, Multiples Races/Two or
More). This project will address what difference in performance progress
were seen between public and charter schools and its respective
population subgroups, as described above. The outcome from this will
give insight to California on the performance progress in order to help
lawmakers better fund schools.

\begin{itemize}
\tightlist
\item
  Is one subgroup out/underperforming another?
\item
  How do districts compare performance wise? County Offices of
  Education? What about with subgroups?
\item
  Charter v. Traditional schools? What about with subgroups?
\item
  Any impacts/trends by \# of kids tested?
\end{itemize}

Data is important because - It tells you how students performed on the
California Assessment of Student Performance and Progress (CAASPP),
california's state wide test - It tells you what achievement/performance
gaps exist - so policy can be made to address these issues

    \section{Data Meaning Type (10)}\label{data-meaning-type-10}

\begin{itemize}
\tightlist
\item
  cds (int): The 14-digit County-District-School code is the official,
  unique identification of a school within California. The first two
  digits identify the county, the next five digits identify the school
  district, and the last seven digits identify the school.
\item
  rtype (char): Identify the record's type;

  \begin{itemize}
  \tightlist
  \item
    S=School Record
  \item
    D=District/LEA Record
  \item
    X=State Record.
  \end{itemize}
\item
  schoolname (str): Name of school the record belongs to.
\item
  districtname (str): Name of school distritct the record belongs to.
\item
  countyname (str): Name of county the record belongs to.
\item
  charter\_flag (char): Indicate if record belongs to a charter school;

  \begin{itemize}
  \tightlist
  \item
    Y=Yes
  \item
    blank
  \end{itemize}
\item
  coe\_flag (char): Indicate if record belongs to a County Office of
  Education facility

  \begin{itemize}
  \tightlist
  \item
    Y=Yes
  \item
    blank
  \end{itemize}
\item
  dass\_flag (char): Indicate if record belongs to a school with a
  Dashboard Alternative School Status (DASS)

  \begin{itemize}
  \tightlist
  \item
    Y=Yes
  \item
    blank
  \end{itemize}
\item
  studentgroup (str): Name of student group the record reflects;

  \begin{itemize}
  \tightlist
  \item
    ALL=All Students
  \item
    AA=Black/African American
  \item
    AI=American Indian or Alaska Native
  \item
    AS=Asian
  \item
    FI=Filipino
  \item
    HI=Hispanic
  \item
    PI=Pacific Islander
  \item
    WH=White
  \item
    MR=Multiple Races/Two or More
  \item
    EL=English Learner
  \item
    ELO=English Learners Only
  \item
    RFP=RFEPs Only
  \item
    EO=English Only
  \item
    SED=Socioeconomically Disadvantaged
  \item
    SWD=Students with Disabilities
  \item
    FOS=Foster Youth
  \item
    HOM=Homeless Youth
  \end{itemize}
\item
  currdenom (int): Number of valid students of the current year who took
  the Smarter Balanced summative assessment in Math. For SWD student
  group only, students who have a current disability code in CALPADS or
  was a student with disability who exited a special education program
  and did not receive services for up to two years (i.e., exited after
  April 15, 2016). These students' scores are used to calculate the
  Distance from Standard.
\item
  currdenom\_swd (int): number of valid students of the current year
  with disabilities who took the Smarter Balanced summative assessment
  in Math and have a current disability code in CALPADS. These students
  and are included in the count to determine whether the SWD student
  group is assigned a performance level. (Note: This count of students
  does not include those who exited a special education program within
  the past two years (i.e., exited program after April 15, 2016).
\item
  currstatus (float): Average distance from Standard of students from
  the current year who took the Smarter Balanced summative assessment in
  Math.
\item
  priordenom (float): Number of valid students of the previous year who
  took the Smarter Balanced summative assessment in Math. For SWD
  student group only, students who have a current disability code in
  CALPADS or was a student with disability who exited a special
  education program and did not receive services for up to two years
  (i.e., exited after April 15, 2016). These students' scores are used
  to calculate the Distance from Standard.
\item
  priordenom\_swd (float): number of valid students of the previous year
  with disabilities who took the Smarter Balanced summative assessment
  in Math and have a current disability code in CALPADS. These students
  and are included in the count to determine whether the SWD student
  group is assigned a performance level. (Note: This count of students
  does not include those who exited a special education program within
  the past two years (i.e., exited program after April 15, 2016).
\item
  priorstatus (float): Average distance from Standard of students from
  the previous year who took the Smarter Balanced summative assessment
  in Math.
\item
  change (float): Difference between current status and prior status.
\item
  statuslevel (int): The LEA, school, or student group's current year a
  Status level for mathematics.

  \begin{itemize}
  \tightlist
  \item
    1=Very Low
  \item
    2=Low
  \item
    3=Medium
  \item
    4=High
  \item
    5=Very High
  \item
    0=No Data
  \end{itemize}
\item
  changelevel (int): The difference in results from the current year to
  the prior year.

  \begin{itemize}
  \tightlist
  \item
    1=Decreased Significantly
  \item
    2=Decreased
  \item
    3=Maintained
  \item
    4=Increased
  \item
    5=Increased Significantly
  \item
    0=No Data
  \end{itemize}
\item
  color (int): The combination of the five Status levels and the five
  Change levels, with red being the lowest and blue the highest in
  performance.

  \begin{itemize}
  \tightlist
  \item
    1=Red
  \item
    2=Orange
  \item
    3=Yellow
  \item
    4=Green
  \item
    5=Blue
  \item
    0=No Color
  \end{itemize}
\item
  box (int): If a color was assigned, this value represents the specific
  5x5 grid location.

  \begin{itemize}
  \tightlist
  \item
    10=Very High, Declined Significantly
  \item
    20=Very High, Declined
  \item
    30=Very High, Mantained
  \item
    40=Very High, Increased
  \item
    50=Very High, Increased Significantly
  \item
    60=High, Declined Significantly
  \item
    70=High, Declined
  \item
    80=High, Mantained
  \item
    90=High, Increased
  \item
    100=High, Increased Significantly
  \item
    110=Medium, Declined Significantly
  \item
    120=Medium, Declined
  \item
    130=Medium, Mantained
  \item
    140=Medium, Increased
  \item
    150=Medium, Increased Significantly
  \item
    160=Low, Declined Significantly
  \item
    170=Low, Declined
  \item
    180=Low, Mantained
  \item
    190=Low, Increased
  \item
    200=Low, Increased Significantly
  \item
    210=Very Low, Declined Significantly
  \item
    220=Very Low, Declined
  \item
    230=Very Low, Mantained
  \item
    240=Very Low, Increased
  \item
    250=Very Low, Increased Significantly
  \item
    0=No Color
  \end{itemize}
\item
  hscutpoints (char): Indicates if the school or district is held to the
  High School cut points.

  \begin{itemize}
  \tightlist
  \item
    Y=Yes
  \item
    blank
  \end{itemize}
\item
  curradjustment (float): The number of points removed from the current
  year status due to the participation rate being below 95\%.
\item
  prioradjustment (float): The number of points removed from the
  previous year status due to the participation rate being below 95\%.
\item
  pairshare\_method (str):This identifies schools that do not serve
  grade 3 or higher and were assigned a district average, county
  average, or weighted school average of up to three receiving schools
  for Distance From Standard.

  \begin{itemize}
  \tightlist
  \item
    DA=District Average
  \item
    CA=County Average
  \item
    WA=Weighted Average
  \item
    blank
  \end{itemize}
\item
  caa\_denom (float): Number of valid students of the current year who
  took the California Alternate Assessment in Math.

  \begin{itemize}
  \tightlist
  \item
    Only populated for records where studentgroup=ALL. If the school or
    district did not test any students under the California Alternate
    Assessment, this value will be zero.
  \end{itemize}
\item
  caa\_level1\_num (float): Number of valid students of the current year
  who took the California Alternate Assessment in Math and scored
  achievement level 1.

  \begin{itemize}
  \tightlist
  \item
    Only populated for records where studentgroup=ALL and
    caa\_denom\textgreater{}=11.
  \end{itemize}
\item
  caa\_level1\_pct (float): Percent of valid students of the current
  year who took the California Alternate Assessment in Math and scored
  achievement level 1.

  \begin{itemize}
  \tightlist
  \item
    Only populated for records where studentgroup=ALL and
    caa\_denom\textgreater{}=11.
  \end{itemize}
\item
  caa\_level2\_num (float): Number of valid students of the current year
  who took the California Alternate Assessment in Math and scored
  achievement level 2.

  \begin{itemize}
  \tightlist
  \item
    Only populated for records where studentgroup=ALL and
    caa\_denom\textgreater{}=11.
  \end{itemize}
\item
  caa\_level2\_pct (float): Percent of valid students of the current
  year who took the California Alternate Assessment in Math and scored
  achievement level 2.

  \begin{itemize}
  \tightlist
  \item
    Only populated for records where studentgroup=ALL and
    caa\_denom\textgreater{}=11.
  \end{itemize}
\item
  caa\_level3\_num (float): Number of valid students of the current year
  who took the California Alternate Assessment in Math and scored
  achievement level 3.

  \begin{itemize}
  \tightlist
  \item
    Only populated for records where studentgroup=ALL and
    caa\_denom\textgreater{}=11.
  \end{itemize}
\item
  caa\_level3\_pct (float): Percent of valid students of the current
  year who took the California Alternate Assessment in Math and scored
  achievement level 3.

  \begin{itemize}
  \tightlist
  \item
    Only populated for records where studentgroup=ALL and
    caa\_denom\textgreater{}=11.
  \end{itemize}
\item
  ReportingYear (int): Reporting year, 2018.
\end{itemize}

    \begin{Verbatim}[commandchars=\\\{\}]
{\color{incolor}In [{\color{incolor}58}]:} \PY{k+kn}{import} \PY{n+nn}{pandas} \PY{k}{as} \PY{n+nn}{pd}
         \PY{k+kn}{import} \PY{n+nn}{numpy} \PY{k}{as} \PY{n+nn}{np}
         \PY{k+kn}{import} \PY{n+nn}{matplotlib}\PY{n+nn}{.}\PY{n+nn}{pyplot} \PY{k}{as} \PY{n+nn}{plt}
         \PY{k+kn}{import} \PY{n+nn}{warnings}
         \PY{n}{warnings}\PY{o}{.}\PY{n}{simplefilter}\PY{p}{(}\PY{l+s+s1}{\PYZsq{}}\PY{l+s+s1}{ignore}\PY{l+s+s1}{\PYZsq{}}\PY{p}{,} \PY{n+ne}{DeprecationWarning}\PY{p}{)}
         \PY{o}{\PYZpc{}}\PY{k}{matplotlib} inline
         
         \PY{c+c1}{\PYZsh{}Import data from data file and name it math2018}
         
         \PY{n}{excel\PYZus{}file} \PY{o}{=} \PY{l+s+s1}{\PYZsq{}}\PY{l+s+s1}{mathdownload2018.xlsx}\PY{l+s+s1}{\PYZsq{}}
         \PY{n}{math2018}\PY{o}{=}\PY{n}{pd}\PY{o}{.}\PY{n}{read\PYZus{}excel}\PY{p}{(}\PY{n}{excel\PYZus{}file}\PY{p}{)}
\end{Verbatim}


    \section{Data Quality (15)}\label{data-quality-15}

At first glance, you will notice that the `All' category total in any
line is less adding up all the lines for similar campuses. This is
happening because a student may be in several different categories
(i.e., a student may be White (WH), in foster care (FOS), be living in a
low socio-economic status (SED) and be served by special education (SWD)
-- student will `count' in 4 of the 16 categories listed). No edits were
needed for this phenomenon, we need to be aware of this as we move
forward.

    \begin{Verbatim}[commandchars=\\\{\}]
{\color{incolor}In [{\color{incolor}59}]:} \PY{c+c1}{\PYZsh{}View basic info about the data file}
         
         \PY{n}{math2018}\PY{o}{.}\PY{n}{info}\PY{p}{(}\PY{p}{)}
\end{Verbatim}


    \begin{Verbatim}[commandchars=\\\{\}]
<class 'pandas.core.frame.DataFrame'>
RangeIndex: 148933 entries, 0 to 148932
Data columns (total 32 columns):
cds                 148933 non-null int64
rtype               148933 non-null object
schoolname          134824 non-null object
districtname        148916 non-null object
countyname          148916 non-null object
charter\_flag        15685 non-null object
coe\_flag            487 non-null object
dass\_flag           7804 non-null object
studentgroup        148933 non-null object
currdenom           148933 non-null int64
currdenom\_swd       10205 non-null float64
currstatus          97487 non-null float64
priordenom          148842 non-null float64
priordenom\_swd      9952 non-null float64
priorstatus         96934 non-null float64
change              93435 non-null float64
statuslevel         148933 non-null int64
changelevel         148933 non-null int64
color               148933 non-null int64
box                 148933 non-null int64
hscutpoints         25661 non-null object
curradjustment      8577 non-null float64
prioradjustment     9795 non-null float64
pairshare\_method    91 non-null object
caa\_denom           10781 non-null float64
caa\_level1\_num      1411 non-null float64
caa\_level1\_pct      1411 non-null float64
caa\_level2\_num      1411 non-null float64
caa\_level2\_pct      1411 non-null float64
caa\_level3\_num      1411 non-null float64
caa\_level3\_pct      1411 non-null float64
ReportingYear       148933 non-null int64
dtypes: float64(15), int64(7), object(10)
memory usage: 36.4+ MB

    \end{Verbatim}

    \begin{Verbatim}[commandchars=\\\{\}]
{\color{incolor}In [{\color{incolor}60}]:} \PY{c+c1}{\PYZsh{}View initial findings of the \PYZdq{}ALL\PYZdq{} cateogry equaly all the other categories from the same campus.}
         
         \PY{n}{math2018}\PY{p}{[}\PY{p}{[}\PY{l+s+s1}{\PYZsq{}}\PY{l+s+s1}{cds}\PY{l+s+s1}{\PYZsq{}}\PY{p}{,} \PY{l+s+s1}{\PYZsq{}}\PY{l+s+s1}{studentgroup}\PY{l+s+s1}{\PYZsq{}}\PY{p}{,} \PY{l+s+s1}{\PYZsq{}}\PY{l+s+s1}{currdenom}\PY{l+s+s1}{\PYZsq{}}\PY{p}{]}\PY{p}{]}
\end{Verbatim}


\begin{Verbatim}[commandchars=\\\{\}]
{\color{outcolor}Out[{\color{outcolor}60}]:}                    cds studentgroup  currdenom
         0                    0           AA     171493
         1                    0           AI      16622
         2                    0          ALL    3166312
         3                    0           AS     293068
         4                    0           EL     999885
         5                    0          ELO     512332
         6                    0           EO    1790552
         7                    0           FI      71096
         8                    0          FOS      20913
         9                    0           HI    1735972
         10                   0          HOM     115267
         11                   0           MR     118473
         12                   0           PI      14849
         13                   0          RFP     487553
         14                   0          SED    1978891
         15                   0          SWD     396607
         16                   0           WH     733191
         17       1100170000000           AA          2
         18       1100170000000          ALL         20
         19       1100170000000           AS          1
         20       1100170000000           EL          8
         21       1100170000000          ELO          8
         22       1100170000000           EO          8
         23       1100170000000           FI          1
         24       1100170000000          FOS          1
         25       1100170000000           HI         14
         26       1100170000000          HOM          2
         27       1100170000000           MR          2
         28       1100170000000          SED         16
         29       1100170000000          SWD          5
         {\ldots}                {\ldots}          {\ldots}        {\ldots}
         148903  58727690000000           EL          6
         148904  58727690000000          ELO          4
         148905  58727690000000           EO        129
         148906  58727690000000           FI          2
         148907  58727690000000          FOS          1
         148908  58727690000000           HI         42
         148909  58727690000000          HOM          1
         148910  58727690000000           MR         18
         148911  58727690000000           PI          1
         148912  58727690000000          RFP          2
         148913  58727690000000          SED         67
         148914  58727690000000          SWD         17
         148915  58727690000000           WH         73
         148916  58727695838305           AA          9
         148917  58727695838305           AI          1
         148918  58727695838305          ALL        152
         148919  58727695838305           AS          4
         148920  58727695838305           EL          6
         148921  58727695838305          ELO          4
         148922  58727695838305           EO        129
         148923  58727695838305           FI          2
         148924  58727695838305          FOS          1
         148925  58727695838305           HI         42
         148926  58727695838305          HOM          1
         148927  58727695838305           MR         18
         148928  58727695838305           PI          1
         148929  58727695838305          RFP          2
         148930  58727695838305          SED         67
         148931  58727695838305          SWD         17
         148932  58727695838305           WH         73
         
         [148933 rows x 3 columns]
\end{Verbatim}
            
    Additionally, there were columns of data that we choose to eliminate
from the data altogether. There were 7 columns at the end of the dataset
that give the data for the ``California Alternate Assessment''. This
test is usually given to students served by special education who have
profound disabilities and are unable to complete the same test as
others. There are approximately 1\% of the total population who are
included, and the data does not support our current research, therefore,
these columns are also deleted.

    \begin{Verbatim}[commandchars=\\\{\}]
{\color{incolor}In [{\color{incolor}61}]:} \PY{c+c1}{\PYZsh{}Delete most of the ending columns that are all about alternate testing: profound disabilities (all the \PYZsq{}caa\PYZus{}...\PYZsq{} columns)}
         \PY{c+c1}{\PYZsh{}Delete reporting year (not needed since all data is 2018)}
         
         \PY{n}{math2018}\PY{o}{.}\PY{n}{drop}\PY{p}{(}\PY{p}{[}\PY{l+s+s1}{\PYZsq{}}\PY{l+s+s1}{caa\PYZus{}denom}\PY{l+s+s1}{\PYZsq{}}\PY{p}{,} \PY{l+s+s1}{\PYZsq{}}\PY{l+s+s1}{caa\PYZus{}level1\PYZus{}num}\PY{l+s+s1}{\PYZsq{}}\PY{p}{,} \PY{l+s+s1}{\PYZsq{}}\PY{l+s+s1}{caa\PYZus{}level1\PYZus{}pct}\PY{l+s+s1}{\PYZsq{}}\PY{p}{,} \PY{l+s+s1}{\PYZsq{}}\PY{l+s+s1}{caa\PYZus{}level2\PYZus{}num}\PY{l+s+s1}{\PYZsq{}}\PY{p}{,} \PY{l+s+s1}{\PYZsq{}}\PY{l+s+s1}{caa\PYZus{}level2\PYZus{}pct}\PY{l+s+s1}{\PYZsq{}}\PY{p}{,} \PY{l+s+s1}{\PYZsq{}}\PY{l+s+s1}{caa\PYZus{}level3\PYZus{}num}\PY{l+s+s1}{\PYZsq{}}\PY{p}{,} \PY{l+s+s1}{\PYZsq{}}\PY{l+s+s1}{caa\PYZus{}level3\PYZus{}pct}\PY{l+s+s1}{\PYZsq{}}\PY{p}{,} \PY{l+s+s1}{\PYZsq{}}\PY{l+s+s1}{ReportingYear}\PY{l+s+s1}{\PYZsq{}}\PY{p}{]}\PY{p}{,} \PY{n}{axis}\PY{o}{=}\PY{l+m+mi}{1}\PY{p}{,} \PY{n}{inplace}\PY{o}{=}\PY{k+kc}{True}\PY{p}{)}
\end{Verbatim}


    Next, we found a column of data called ``pairshare\_method''. If a
campus does not include students of a testing grade level (i.e. PK
campuses) or the campus only serves special populations (i.e., mental
health hospital), the data is often paired with another campus in the
district or charter. If we kept this data in our dataset, these would be
duplicates. Therefore, the lines of data were deleted and then the
column was deleted.

    \begin{Verbatim}[commandchars=\\\{\}]
{\color{incolor}In [{\color{incolor}62}]:} \PY{c+c1}{\PYZsh{}We can also delete rows of data for campuses that are pairshare\PYZus{}method (\PYZsq{}CA\PYZsq{}, \PYZsq{}DA\PYZsq{}, \PYZsq{}WA\PYZsq{})}
         \PY{c+c1}{\PYZsh{}there are 91 rows of data we can remove this way.}
         \PY{c+c1}{\PYZsh{}Pairshare means that their scores were shared with another campus}
         
         \PY{n}{math2018}\PY{o}{.}\PY{n}{drop}\PY{p}{(}\PY{n}{math2018}\PY{p}{[}\PY{p}{(}\PY{n}{math2018}\PY{o}{.}\PY{n}{pairshare\PYZus{}method} \PY{o}{==} \PY{l+s+s1}{\PYZsq{}}\PY{l+s+s1}{CA}\PY{l+s+s1}{\PYZsq{}}\PY{p}{)} \PY{o}{|} \PY{p}{(}\PY{n}{math2018}\PY{o}{.}\PY{n}{pairshare\PYZus{}method} \PY{o}{==} \PY{l+s+s1}{\PYZsq{}}\PY{l+s+s1}{DA}\PY{l+s+s1}{\PYZsq{}}\PY{p}{)} \PY{o}{|} \PY{p}{(}\PY{n}{math2018}\PY{o}{.}\PY{n}{pairshare\PYZus{}method} \PY{o}{==} \PY{l+s+s1}{\PYZsq{}}\PY{l+s+s1}{WA}\PY{l+s+s1}{\PYZsq{}}\PY{p}{)}\PY{p}{]}\PY{o}{.}\PY{n}{index}\PY{p}{,} \PY{n}{inplace}\PY{o}{=}\PY{k+kc}{True}\PY{p}{)}
         
         \PY{c+c1}{\PYZsh{}Now that those lines have been deleted, we can eliminate the pairshare\PYZus{}method }
         
         \PY{n}{math2018}\PY{o}{.}\PY{n}{drop}\PY{p}{(}\PY{p}{[}\PY{l+s+s1}{\PYZsq{}}\PY{l+s+s1}{pairshare\PYZus{}method}\PY{l+s+s1}{\PYZsq{}}\PY{p}{]}\PY{p}{,} \PY{n}{axis}\PY{o}{=}\PY{l+m+mi}{1}\PY{p}{,} \PY{n}{inplace}\PY{o}{=}\PY{k+kc}{True}\PY{p}{)}
\end{Verbatim}


    There are many missing values, however, they are not mistakes. The
values that are missing appear when a population has 10 or fewer
students represented. When this occurs, the data is not calculated for
the purpose of this data file and state accountability ratings. As we
look at the data, you will notice some of the larger category averages
come out to NaN (Not a Number) because each individual total is 10 or
fewer. We chose to delete these lines from our dataset because the data
we need to calculate is not included for these lines.

    \begin{Verbatim}[commandchars=\\\{\}]
{\color{incolor}In [{\color{incolor}63}]:} \PY{c+c1}{\PYZsh{}Delete any row of data where there is no Currstatus data \PYZhy{} meaning that there were less than 10 students that make up that group}
         
         \PY{n}{math2018}\PY{o}{.}\PY{n}{dropna}\PY{p}{(}\PY{n}{subset}\PY{o}{=}\PY{p}{[}\PY{l+s+s2}{\PYZdq{}}\PY{l+s+s2}{currstatus}\PY{l+s+s2}{\PYZdq{}}\PY{p}{]}\PY{p}{,} \PY{n}{inplace}\PY{o}{=}\PY{k+kc}{True}\PY{p}{)}
\end{Verbatim}


    \begin{Verbatim}[commandchars=\\\{\}]
{\color{incolor}In [{\color{incolor}64}]:} \PY{c+c1}{\PYZsh{}Perform a \PYZsq{}sanity check\PYZsq{} to ensure we took out the data that needed to be reomved}
         
         \PY{n}{math2018}\PY{o}{.}\PY{n}{info}\PY{p}{(}\PY{p}{)}
\end{Verbatim}


    \begin{Verbatim}[commandchars=\\\{\}]
<class 'pandas.core.frame.DataFrame'>
Int64Index: 97396 entries, 0 to 148932
Data columns (total 23 columns):
cds                97396 non-null int64
rtype              97396 non-null object
schoolname         86927 non-null object
districtname       97379 non-null object
countyname         97379 non-null object
charter\_flag       9498 non-null object
coe\_flag           154 non-null object
dass\_flag          2220 non-null object
studentgroup       97396 non-null object
currdenom          97396 non-null int64
currdenom\_swd      8713 non-null float64
currstatus         97396 non-null float64
priordenom         97396 non-null float64
priordenom\_swd     8662 non-null float64
priorstatus        93344 non-null float64
change             93344 non-null float64
statuslevel        97396 non-null int64
changelevel        97396 non-null int64
color              97396 non-null int64
box                97396 non-null int64
hscutpoints        13804 non-null object
curradjustment     8577 non-null float64
prioradjustment    8902 non-null float64
dtypes: float64(8), int64(6), object(9)
memory usage: 17.8+ MB

    \end{Verbatim}

    \section{Simple Statistics (10)}\label{simple-statistics-10}

We ran df.describe() on the full dataset to understand the range,
average, and quartile of the numerical values. Due to the way the data
is setup, this doesn't give us very much information. From this output,
however, we can use the `count' to find out the location of most of our
`NaN's and ensure they most were removed or justified during our work on
data quality above.

    \begin{Verbatim}[commandchars=\\\{\}]
{\color{incolor}In [{\color{incolor}65}]:} \PY{c+c1}{\PYZsh{}Explore simple statistics}
         
         \PY{n}{math2018}\PY{o}{.}\PY{n}{describe}\PY{p}{(}\PY{p}{)}
\end{Verbatim}


\begin{Verbatim}[commandchars=\\\{\}]
{\color{outcolor}Out[{\color{outcolor}65}]:}                 cds     currdenom  currdenom\_swd    currstatus    priordenom  \textbackslash{}
         count  9.739600e+04  9.739600e+04    8713.000000  97396.000000  9.739600e+04   
         mean   2.943364e+13  3.636016e+02     115.593481    -50.096683  3.656199e+02   
         std    1.395610e+13  1.526794e+04    3886.330054     66.741800  1.535247e+04   
         min    0.000000e+00  1.100000e+01       1.000000   -298.500000  0.000000e+00   
         25\%    1.964733e+13  3.100000e+01      24.000000    -90.000000  3.000000e+01   
         50\%    3.073644e+13  7.700000e+01      36.000000    -48.900000  7.700000e+01   
         75\%    3.773570e+13  1.990000e+02      55.000000     -6.100000  2.020000e+02   
         max    5.872770e+13  3.166312e+06  361212.000000    178.900000  3.195814e+06   
         
                priordenom\_swd   priorstatus        change   statuslevel   changelevel  \textbackslash{}
         count     8662.000000  93344.000000  93344.000000  97396.000000  97396.000000   
         mean       113.694181    -51.016605      1.688386      2.479886      3.028564   
         std       3812.808540     64.409699     18.125985      1.236779      1.443583   
         min          1.000000   -289.900000   -192.700000      1.000000      0.000000   
         25\%         23.000000    -89.100000     -7.500000      2.000000      2.000000   
         50\%         35.000000    -50.700000      2.000000      2.000000      3.000000   
         75\%         55.000000     -9.100000     11.400000      3.000000      4.000000   
         max     353367.000000    169.900000    191.900000      5.000000      5.000000   
         
                       color           box  curradjustment  prioradjustment  
         count  97396.000000  97396.000000     8577.000000      8902.000000  
         mean       1.491458     83.455686       -1.870613        -1.772551  
         std        1.641152     89.738640        2.466124         2.289465  
         min        0.000000      0.000000      -20.500000       -21.750000  
         25\%        0.000000      0.000000       -2.250000        -2.000000  
         50\%        1.000000     40.000000       -1.000000        -1.000000  
         75\%        3.000000    180.000000       -0.500000        -0.500000  
         max        5.000000    250.000000       -0.250000        -0.250000  
\end{Verbatim}
            
    We also created a pivot table with our 3 columns of interest as we move
forward. The dataline at '0' is the state totals. It is clear that as a
state, California is scoring at 51 points below the passing standard,
however, they have improved their score by 1 point as compared to last
year.

    \begin{Verbatim}[commandchars=\\\{\}]
{\color{incolor}In [{\color{incolor}66}]:} \PY{c+c1}{\PYZsh{}Making the above more meaningful by only including attributes of interest}
         
         \PY{n}{math2018}\PY{o}{.}\PY{n}{pivot\PYZus{}table}\PY{p}{(}\PY{n}{index}\PY{o}{=}\PY{l+s+s1}{\PYZsq{}}\PY{l+s+s1}{cds}\PY{l+s+s1}{\PYZsq{}}\PY{p}{,} \PY{n}{values} \PY{o}{=} \PY{p}{[}\PY{l+s+s1}{\PYZsq{}}\PY{l+s+s1}{currdenom}\PY{l+s+s1}{\PYZsq{}}\PY{p}{,} \PY{l+s+s1}{\PYZsq{}}\PY{l+s+s1}{currstatus}\PY{l+s+s1}{\PYZsq{}}\PY{p}{,} \PY{l+s+s1}{\PYZsq{}}\PY{l+s+s1}{change}\PY{l+s+s1}{\PYZsq{}}\PY{p}{]}\PY{p}{,} \PY{n}{aggfunc}\PY{o}{=}\PY{l+s+s1}{\PYZsq{}}\PY{l+s+s1}{mean}\PY{l+s+s1}{\PYZsq{}}\PY{p}{)}
\end{Verbatim}


\begin{Verbatim}[commandchars=\\\{\}]
{\color{outcolor}Out[{\color{outcolor}66}]:}                    change      currdenom  currstatus
         cds                                                 
         0                1.064706  742533.882353  -51.100000
         1100170000000   13.000000      16.666667 -242.500000
         1100170112607   13.728571      45.285714 -105.114286
         1100170123968  -10.425000      59.000000  -88.362500
         1100170124172   21.900000      66.222222   81.055556
         1100170125567    4.133333      58.083333  -56.258333
         1100170130419         NaN      14.000000 -223.300000
         1100170131581   16.466667      83.111111 -114.722222
         1100170136101         NaN      37.333333   23.900000
         1100176001788   12.222222     130.777778  -69.011111
         1100176002000   43.122222     145.666667  -51.177778
         1611190000000    1.268750    1017.062500  -23.906250
         1611190106401   30.475000      23.750000   90.725000
         1611190111765    3.827273      84.454545  -63.281818
         1611190119222   10.350000      81.166667  -57.458333
         1611190122085  -13.207692     133.230769  -25.123077
         1611190126656   -2.291667      50.833333  -37.083333
         1611190130229    7.369231      95.307692  -22.553846
         1611190130609    2.858333      53.083333  -11.408333
         1611190130625  -21.866667      13.666667 -222.533333
         1611190131805         NaN      24.000000  -16.514286
         1611190132142   13.323077     139.384615  -48.600000
         1611190134304   16.900000      21.500000 -163.875000
         1611196090005  -10.510000      31.200000   31.100000
         1611196090013    0.172727      68.727273   25.109091
         1611196090021    0.654545      81.500000    3.325000
         1611196090039   -5.600000      56.900000   24.160000
         1611196090047   12.407692      80.384615  -11.961538
         1611196090054   -5.884615     214.461538    4.523077
         1611196090112    8.269231     142.461538  -39.576923
         {\ldots}                   {\ldots}            {\ldots}         {\ldots}
         58727365830096  -5.466667      20.500000 -171.650000
         58727365830138 -10.071429      87.000000  -44.437500
         58727365835202  19.287500      69.222222 -127.733333
         58727366056626   2.400000     150.928571  -81.400000
         58727366056634  -5.036364      97.454545  -39.227273
         58727366056659  -1.255556     114.700000  -82.590000
         58727366056667   4.400000      30.800000  -43.860000
         58727366056675   4.150000      17.750000  -92.500000
         58727366056683  -4.260000     135.800000  -42.870000
         58727366056691  16.550000      86.000000  -61.112500
         58727366056709   2.740000      81.200000  -54.570000
         58727366056717  13.790000     136.727273  -63.645455
         58727366056725  -6.160000      33.800000  -33.640000
         58727366056733  -0.022222      83.700000    0.130000
         58727366056741   1.510000     110.000000  -73.550000
         58727366056774   4.230000     115.583333  -25.316667
         58727366056782   0.516667      34.833333  -76.216667
         58727366056790   4.561538     235.615385 -116.976923
         58727366099014  -8.775000      56.250000   14.975000
         58727440000000  14.225000     225.583333  -26.816667
         58727440107375  12.012500      69.500000    9.210000
         58727440112623  15.954545     125.636364  -50.500000
         58727440114652  26.175000      65.666667  -11.466667
         58727510000000   7.875000     194.500000  -44.128571
         58727516056816  14.808333     150.230769  -47.430769
         58727516056832  -4.250000      62.000000  -28.062500
         58727516056840 -18.033333      37.166667  -33.366667
         58727516118806   4.200000      17.000000    1.633333
         58727690000000 -10.200000      71.142857  -95.800000
         58727695838305 -11.720000      71.142857  -95.700000
         
         [10093 rows x 3 columns]
\end{Verbatim}
            
    \section{Explore Joint Attributes
(15)}\label{explore-joint-attributes-15}

In order to get a better look at the SchoolType and the studentgroup, we
created a new attribute, School Type. After doing so, we used groupby to
create a group-wise average. Once the new column, called "SchoolType"
was created, the former columns were deleted.

    \begin{Verbatim}[commandchars=\\\{\}]
{\color{incolor}In [{\color{incolor}67}]:} \PY{c+c1}{\PYZsh{}Create a new column, SchoolType; label each row with an appropriate label: state, county, district, public, charter, alternative public, and alternative charter}
         
         \PY{n}{math2018}\PY{o}{.}\PY{n}{loc}\PY{p}{[}\PY{p}{(}\PY{n}{math2018}\PY{o}{.}\PY{n}{charter\PYZus{}flag} \PY{o}{==} \PY{l+s+s1}{\PYZsq{}}\PY{l+s+s1}{Y}\PY{l+s+s1}{\PYZsq{}}\PY{p}{)} \PY{o}{\PYZam{}} \PY{p}{(}\PY{n}{math2018}\PY{o}{.}\PY{n}{dass\PYZus{}flag} \PY{o}{!=} \PY{l+s+s1}{\PYZsq{}}\PY{l+s+s1}{Y}\PY{l+s+s1}{\PYZsq{}}\PY{p}{)}\PY{p}{,} \PY{l+s+s1}{\PYZsq{}}\PY{l+s+s1}{SchoolType}\PY{l+s+s1}{\PYZsq{}}\PY{p}{]} \PY{o}{=} \PY{l+s+s1}{\PYZsq{}}\PY{l+s+s1}{Charter}\PY{l+s+s1}{\PYZsq{}}
         \PY{n}{math2018}\PY{o}{.}\PY{n}{loc}\PY{p}{[}\PY{p}{(}\PY{n}{math2018}\PY{o}{.}\PY{n}{charter\PYZus{}flag} \PY{o}{==} \PY{l+s+s1}{\PYZsq{}}\PY{l+s+s1}{Y}\PY{l+s+s1}{\PYZsq{}}\PY{p}{)} \PY{o}{\PYZam{}} \PY{p}{(}\PY{n}{math2018}\PY{o}{.}\PY{n}{dass\PYZus{}flag} \PY{o}{==} \PY{l+s+s1}{\PYZsq{}}\PY{l+s+s1}{Y}\PY{l+s+s1}{\PYZsq{}}\PY{p}{)}\PY{p}{,} \PY{l+s+s1}{\PYZsq{}}\PY{l+s+s1}{SchoolType}\PY{l+s+s1}{\PYZsq{}}\PY{p}{]} \PY{o}{=} \PY{l+s+s1}{\PYZsq{}}\PY{l+s+s1}{AltCharter}\PY{l+s+s1}{\PYZsq{}}
         \PY{n}{math2018}\PY{o}{.}\PY{n}{loc}\PY{p}{[}\PY{n}{math2018}\PY{o}{.}\PY{n}{coe\PYZus{}flag} \PY{o}{==} \PY{l+s+s1}{\PYZsq{}}\PY{l+s+s1}{Y}\PY{l+s+s1}{\PYZsq{}}\PY{p}{,} \PY{l+s+s1}{\PYZsq{}}\PY{l+s+s1}{SchoolType}\PY{l+s+s1}{\PYZsq{}}\PY{p}{]} \PY{o}{=} \PY{l+s+s1}{\PYZsq{}}\PY{l+s+s1}{County}\PY{l+s+s1}{\PYZsq{}}
         \PY{n}{math2018}\PY{o}{.}\PY{n}{loc}\PY{p}{[}\PY{p}{(}\PY{n}{math2018}\PY{o}{.}\PY{n}{dass\PYZus{}flag} \PY{o}{==} \PY{l+s+s1}{\PYZsq{}}\PY{l+s+s1}{Y}\PY{l+s+s1}{\PYZsq{}}\PY{p}{)} \PY{o}{\PYZam{}} \PY{p}{(}\PY{n}{math2018}\PY{o}{.}\PY{n}{charter\PYZus{}flag} \PY{o}{!=} \PY{l+s+s1}{\PYZsq{}}\PY{l+s+s1}{Y}\PY{l+s+s1}{\PYZsq{}}\PY{p}{)}\PY{p}{,} \PY{l+s+s1}{\PYZsq{}}\PY{l+s+s1}{SchoolType}\PY{l+s+s1}{\PYZsq{}}\PY{p}{]} \PY{o}{=} \PY{l+s+s1}{\PYZsq{}}\PY{l+s+s1}{AltPublic}\PY{l+s+s1}{\PYZsq{}}
         \PY{n}{math2018}\PY{o}{.}\PY{n}{loc}\PY{p}{[}\PY{p}{(}\PY{n}{math2018}\PY{o}{.}\PY{n}{rtype} \PY{o}{==} \PY{l+s+s1}{\PYZsq{}}\PY{l+s+s1}{S}\PY{l+s+s1}{\PYZsq{}}\PY{p}{)} \PY{o}{\PYZam{}} \PY{p}{(}\PY{n}{math2018}\PY{o}{.}\PY{n}{charter\PYZus{}flag} \PY{o}{!=} \PY{l+s+s1}{\PYZsq{}}\PY{l+s+s1}{Y}\PY{l+s+s1}{\PYZsq{}}\PY{p}{)} \PY{o}{\PYZam{}} \PY{p}{(}\PY{n}{math2018}\PY{o}{.}\PY{n}{dass\PYZus{}flag} \PY{o}{!=} \PY{l+s+s1}{\PYZsq{}}\PY{l+s+s1}{Y}\PY{l+s+s1}{\PYZsq{}}\PY{p}{)}\PY{p}{,} \PY{l+s+s1}{\PYZsq{}}\PY{l+s+s1}{SchoolType}\PY{l+s+s1}{\PYZsq{}}\PY{p}{]} \PY{o}{=} \PY{l+s+s1}{\PYZsq{}}\PY{l+s+s1}{Public}\PY{l+s+s1}{\PYZsq{}}
         \PY{n}{math2018}\PY{o}{.}\PY{n}{loc}\PY{p}{[}\PY{p}{(}\PY{n}{math2018}\PY{o}{.}\PY{n}{rtype} \PY{o}{==} \PY{l+s+s1}{\PYZsq{}}\PY{l+s+s1}{X}\PY{l+s+s1}{\PYZsq{}}\PY{p}{)}\PY{p}{,} \PY{l+s+s1}{\PYZsq{}}\PY{l+s+s1}{SchoolType}\PY{l+s+s1}{\PYZsq{}}\PY{p}{]} \PY{o}{=} \PY{l+s+s1}{\PYZsq{}}\PY{l+s+s1}{State}\PY{l+s+s1}{\PYZsq{}}
         \PY{n}{math2018}\PY{o}{.}\PY{n}{loc}\PY{p}{[}\PY{p}{(}\PY{n}{math2018}\PY{o}{.}\PY{n}{rtype} \PY{o}{==} \PY{l+s+s1}{\PYZsq{}}\PY{l+s+s1}{D}\PY{l+s+s1}{\PYZsq{}}\PY{p}{)} \PY{o}{\PYZam{}} \PY{p}{(}\PY{n}{math2018}\PY{o}{.}\PY{n}{coe\PYZus{}flag} \PY{o}{!=} \PY{l+s+s1}{\PYZsq{}}\PY{l+s+s1}{Y}\PY{l+s+s1}{\PYZsq{}}\PY{p}{)}\PY{p}{,} \PY{l+s+s1}{\PYZsq{}}\PY{l+s+s1}{SchoolType}\PY{l+s+s1}{\PYZsq{}}\PY{p}{]} \PY{o}{=} \PY{l+s+s1}{\PYZsq{}}\PY{l+s+s1}{District}\PY{l+s+s1}{\PYZsq{}}
         
         \PY{c+c1}{\PYZsh{}Delete columns that are no longer necessary now that the school type is combined (rtype, charter\PYZus{}flag, coe\PYZus{}flag, dass\PYZus{}flag)}
         
         \PY{n}{math2018}\PY{o}{.}\PY{n}{drop}\PY{p}{(}\PY{p}{[}\PY{l+s+s1}{\PYZsq{}}\PY{l+s+s1}{rtype}\PY{l+s+s1}{\PYZsq{}}\PY{p}{,} \PY{l+s+s1}{\PYZsq{}}\PY{l+s+s1}{charter\PYZus{}flag}\PY{l+s+s1}{\PYZsq{}}\PY{p}{,} \PY{l+s+s1}{\PYZsq{}}\PY{l+s+s1}{coe\PYZus{}flag}\PY{l+s+s1}{\PYZsq{}}\PY{p}{,} \PY{l+s+s1}{\PYZsq{}}\PY{l+s+s1}{dass\PYZus{}flag}\PY{l+s+s1}{\PYZsq{}}\PY{p}{]}\PY{p}{,} \PY{n}{axis}\PY{o}{=}\PY{l+m+mi}{1}\PY{p}{,} \PY{n}{inplace}\PY{o}{=}\PY{k+kc}{True}\PY{p}{)}
\end{Verbatim}


    As we explored the data further, we found the need to drop a few more
aspects of our data set. One final step in the initial cleaning process
was to remove the State, County and District lines of data. Leaving
these lines of data in the file, leads to duplicates. Every school
belongs to 3 larger organizations - it's respective district, county,
and state.

    \begin{Verbatim}[commandchars=\\\{\}]
{\color{incolor}In [{\color{incolor}68}]:} \PY{n}{math2018}\PY{o}{.}\PY{n}{drop}\PY{p}{(}\PY{n}{math2018}\PY{p}{[}\PY{p}{(}\PY{n}{math2018}\PY{o}{.}\PY{n}{SchoolType}\PY{o}{==}\PY{l+s+s1}{\PYZsq{}}\PY{l+s+s1}{State}\PY{l+s+s1}{\PYZsq{}}\PY{p}{)} \PY{o}{|} \PY{p}{(}\PY{n}{math2018}\PY{o}{.}\PY{n}{SchoolType}\PY{o}{==}\PY{l+s+s1}{\PYZsq{}}\PY{l+s+s1}{County}\PY{l+s+s1}{\PYZsq{}}\PY{p}{)} \PY{o}{|} \PY{p}{(}\PY{n}{math2018}\PY{o}{.}\PY{n}{SchoolType}\PY{o}{==}\PY{l+s+s1}{\PYZsq{}}\PY{l+s+s1}{District}\PY{l+s+s1}{\PYZsq{}}\PY{p}{)}\PY{p}{]}\PY{o}{.}\PY{n}{index}\PY{p}{,} \PY{n}{inplace}\PY{o}{=}\PY{k+kc}{True}\PY{p}{)}
\end{Verbatim}


    \begin{Verbatim}[commandchars=\\\{\}]
{\color{incolor}In [{\color{incolor}69}]:} \PY{c+c1}{\PYZsh{}Group School Type and Student Group and report average distance from \PYZsq{}passing\PYZsq{} for 2018}
         
         \PY{n}{math2018\PYZus{}grouped}\PY{o}{=} \PY{n}{math2018}\PY{o}{.}\PY{n}{groupby}\PY{p}{(}\PY{n}{by}\PY{o}{=}\PY{p}{[}\PY{l+s+s1}{\PYZsq{}}\PY{l+s+s1}{SchoolType}\PY{l+s+s1}{\PYZsq{}}\PY{p}{,} \PY{l+s+s1}{\PYZsq{}}\PY{l+s+s1}{studentgroup}\PY{l+s+s1}{\PYZsq{}}\PY{p}{]}\PY{p}{)}
         \PY{n}{math2018\PYZus{}AvgDistance} \PY{o}{=} \PY{n}{math2018\PYZus{}grouped}\PY{o}{.}\PY{n}{currstatus}\PY{o}{.}\PY{n}{sum}\PY{p}{(}\PY{p}{)} \PY{o}{/} \PY{n}{math2018\PYZus{}grouped}\PY{o}{.}\PY{n}{currstatus}\PY{o}{.}\PY{n}{count}\PY{p}{(}\PY{p}{)}
         \PY{n}{math2018\PYZus{}AvgDistance}
\end{Verbatim}


\begin{Verbatim}[commandchars=\\\{\}]
{\color{outcolor}Out[{\color{outcolor}69}]:} SchoolType  studentgroup
         AltCharter  AA             -169.726087
                     AI             -131.742857
                     ALL            -162.358824
                     AS              -88.800000
                     EL             -185.802381
                     ELO            -210.133333
                     EO             -154.000000
                     FI              -30.800000
                     HI             -169.618056
                     HOM            -189.800000
                     MR             -117.754545
                     RFP            -159.088000
                     SED            -168.763750
                     SWD            -201.433333
                     WH             -131.386957
         AltPublic   AA             -220.220000
                     ALL            -199.196992
                     AS              -37.725000
                     EL             -222.866923
                     ELO            -225.562500
                     EO             -190.538745
                     FI               37.100000
                     HI             -206.182935
                     HOM            -213.363158
                     MR              -73.733333
                     RFP            -207.442857
                     SED            -205.226048
                     SWD            -196.759091
                     WH             -159.682540
         Charter     AA              -76.660156
                                        {\ldots}    
                     EL              -66.374906
                     ELO            -103.180813
                     EO              -36.011896
                     FI               18.334906
                     FOS            -178.900000
                     HI              -51.828049
                     HOM             -75.477083
                     MR               -7.365854
                     PI              -56.950000
                     RFP             -29.977321
                     SED             -55.791594
                     SWD            -107.658202
                     WH              -14.017647
         Public      AA              -78.072563
                     AI              -81.952318
                     ALL             -32.908113
                     AS               35.992458
                     EL              -57.178947
                     ELO             -97.511655
                     EO              -30.015609
                     FI               15.295572
                     FOS            -123.250000
                     HI              -50.756406
                     HOM             -75.074270
                     MR                8.097910
                     PI              -69.436667
                     RFP             -17.766481
                     SED             -51.550202
                     SWD            -113.291341
                     WH               -9.582096
         Name: currstatus, Length: 63, dtype: float64
\end{Verbatim}
            
    When we looked at these data, we found some significant differences in
the average distance from passing between the different school types and
student groups.

    Additionally, we created a cross-tabulation chart to enable us to
visualize the information in barplot form. Seeing this visually, enabled
us to see the stark differences in the average distance from passing in
each school type.

    \begin{Verbatim}[commandchars=\\\{\}]
{\color{incolor}In [{\color{incolor}70}]:} \PY{c+c1}{\PYZsh{}Average Distance from \PYZsq{}Passing\PYZsq{}}
         
         \PY{n}{pd}\PY{o}{.}\PY{n}{crosstab}\PY{p}{(}\PY{n}{math2018}\PY{o}{.}\PY{n}{SchoolType}\PY{p}{,} \PY{n}{math2018}\PY{o}{.}\PY{n}{studentgroup}\PY{p}{,} \PY{n}{values} \PY{o}{=} \PY{n}{math2018}\PY{o}{.}\PY{n}{currstatus}\PY{p}{,} \PY{n}{aggfunc}\PY{o}{=}\PY{l+s+s1}{\PYZsq{}}\PY{l+s+s1}{mean}\PY{l+s+s1}{\PYZsq{}}\PY{p}{)}\PY{o}{.}\PY{n}{round}\PY{p}{(}\PY{l+m+mi}{2}\PY{p}{)}
\end{Verbatim}


\begin{Verbatim}[commandchars=\\\{\}]
{\color{outcolor}Out[{\color{outcolor}70}]:} studentgroup      AA      AI     ALL     AS      EL     ELO      EO     FI  \textbackslash{}
         SchoolType                                                                   
         AltCharter   -169.73 -131.74 -162.36 -88.80 -185.80 -210.13 -154.00 -30.80   
         AltPublic    -220.22     NaN -199.20 -37.72 -222.87 -225.56 -190.54  37.10   
         Charter       -76.66 -102.71  -40.01  39.17  -66.37 -103.18  -36.01  18.33   
         Public        -78.07  -81.95  -32.91  35.99  -57.18  -97.51  -30.02  15.30   
         
         studentgroup     FOS      HI     HOM      MR     PI     RFP     SED     SWD  \textbackslash{}
         SchoolType                                                                    
         AltCharter       NaN -169.62 -189.80 -117.75    NaN -159.09 -168.76 -201.43   
         AltPublic        NaN -206.18 -213.36  -73.73    NaN -207.44 -205.23 -196.76   
         Charter      -178.90  -51.83  -75.48   -7.37 -56.95  -29.98  -55.79 -107.66   
         Public       -123.25  -50.76  -75.07    8.10 -69.44  -17.77  -51.55 -113.29   
         
         studentgroup      WH  
         SchoolType            
         AltCharter   -131.39  
         AltPublic    -159.68  
         Charter       -14.02  
         Public         -9.58  
\end{Verbatim}
            
    \section{Visualize Attributes (15)}\label{visualize-attributes-15}

We found that the best way, initially to visualize the data was to first
look at the average distance from passing, as caluclated through an
aggregate function.

    \begin{Verbatim}[commandchars=\\\{\}]
{\color{incolor}In [{\color{incolor}71}]:} \PY{c+c1}{\PYZsh{}Group School Type and Student Group and report average distance from \PYZsq{}passing\PYZsq{} for 2018}
         
         \PY{n}{math2018\PYZus{}grouped}\PY{o}{=} \PY{n}{math2018}\PY{o}{.}\PY{n}{groupby}\PY{p}{(}\PY{n}{by}\PY{o}{=}\PY{p}{[}\PY{l+s+s1}{\PYZsq{}}\PY{l+s+s1}{SchoolType}\PY{l+s+s1}{\PYZsq{}}\PY{p}{,} \PY{l+s+s1}{\PYZsq{}}\PY{l+s+s1}{studentgroup}\PY{l+s+s1}{\PYZsq{}}\PY{p}{]}\PY{p}{)}
         \PY{n}{math2018\PYZus{}AvgDistance} \PY{o}{=} \PY{n}{math2018\PYZus{}grouped}\PY{o}{.}\PY{n}{currstatus}\PY{o}{.}\PY{n}{sum}\PY{p}{(}\PY{p}{)} \PY{o}{/} \PY{n}{math2018\PYZus{}grouped}\PY{o}{.}\PY{n}{currstatus}\PY{o}{.}\PY{n}{count}\PY{p}{(}\PY{p}{)}
         \PY{n}{math2018\PYZus{}AvgDistance}
\end{Verbatim}


\begin{Verbatim}[commandchars=\\\{\}]
{\color{outcolor}Out[{\color{outcolor}71}]:} SchoolType  studentgroup
         AltCharter  AA             -169.726087
                     AI             -131.742857
                     ALL            -162.358824
                     AS              -88.800000
                     EL             -185.802381
                     ELO            -210.133333
                     EO             -154.000000
                     FI              -30.800000
                     HI             -169.618056
                     HOM            -189.800000
                     MR             -117.754545
                     RFP            -159.088000
                     SED            -168.763750
                     SWD            -201.433333
                     WH             -131.386957
         AltPublic   AA             -220.220000
                     ALL            -199.196992
                     AS              -37.725000
                     EL             -222.866923
                     ELO            -225.562500
                     EO             -190.538745
                     FI               37.100000
                     HI             -206.182935
                     HOM            -213.363158
                     MR              -73.733333
                     RFP            -207.442857
                     SED            -205.226048
                     SWD            -196.759091
                     WH             -159.682540
         Charter     AA              -76.660156
                                        {\ldots}    
                     EL              -66.374906
                     ELO            -103.180813
                     EO              -36.011896
                     FI               18.334906
                     FOS            -178.900000
                     HI              -51.828049
                     HOM             -75.477083
                     MR               -7.365854
                     PI              -56.950000
                     RFP             -29.977321
                     SED             -55.791594
                     SWD            -107.658202
                     WH              -14.017647
         Public      AA              -78.072563
                     AI              -81.952318
                     ALL             -32.908113
                     AS               35.992458
                     EL              -57.178947
                     ELO             -97.511655
                     EO              -30.015609
                     FI               15.295572
                     FOS            -123.250000
                     HI              -50.756406
                     HOM             -75.074270
                     MR                8.097910
                     PI              -69.436667
                     RFP             -17.766481
                     SED             -51.550202
                     SWD            -113.291341
                     WH               -9.582096
         Name: currstatus, Length: 63, dtype: float64
\end{Verbatim}
            
    \begin{Verbatim}[commandchars=\\\{\}]
{\color{incolor}In [{\color{incolor}72}]:} \PY{n}{ax}\PY{o}{=}\PY{n}{math2018\PYZus{}AvgDistance}\PY{o}{.}\PY{n}{plot}\PY{p}{(}\PY{n}{kind}\PY{o}{=}\PY{l+s+s1}{\PYZsq{}}\PY{l+s+s1}{barh}\PY{l+s+s1}{\PYZsq{}}\PY{p}{)}
\end{Verbatim}


    \begin{center}
    \adjustimage{max size={0.9\linewidth}{0.9\paperheight}}{output_28_0.png}
    \end{center}
    { \hspace*{\fill} \\}
    
    In doing so, we realized that though this visual makes it easy to see
that there is great discrepancey between differen student groups and
school types, the data is not very 'human readable' (For further
discovery, visit Exception Work below).

    \section{Explore Attributes and Class
(10)}\label{explore-attributes-and-class-10}

    \section{New Features (5)}\label{new-features-5}

As mentioned above, the data is presented in such a way that made
comparing different school types difficult. There are various columns
with a 'Y' or blank entry, each based on the data being from either a
Charter School, County Entity, or an Alternative School (3 separate
columns). Additionally, there is another column that tells us if the
data is from either state, district/LEA, or school (1 column). In order
to get one label for each line of data based on this data, a new feature
was created called SchoolType. The New feature created was,
"SchoolType". Based on the 4 columns mentioned above, SchoolType
attributes are Public, Charter, AltPublic, AltCharter, District, County,
and State. (see code from Explore Attributes)

    \section{Exceptional Work (10)}\label{exceptional-work-10}

In order to better visualize the information from the student groups
column, we split the data into two different groupings. One grouping is
all race/ethnicity categories. The other grouping is all of the various
special programs that are measured. The "ALL" category was left out as
it is simply the average of the averages all together. The data below
represent the average distance from passing for each student group and
school type.

    \begin{Verbatim}[commandchars=\\\{\}]
{\color{incolor}In [{\color{incolor}73}]:} \PY{c+c1}{\PYZsh{}Create an Ethnicity/Race Subgroup to compare like data.}
         
         \PY{n}{math2018ETH}\PY{o}{=}\PY{p}{(}\PY{n}{math2018}\PY{p}{[}\PY{p}{(}\PY{n}{math2018}\PY{o}{.}\PY{n}{studentgroup} \PY{o}{==} \PY{l+s+s1}{\PYZsq{}}\PY{l+s+s1}{AA}\PY{l+s+s1}{\PYZsq{}}\PY{p}{)} \PY{o}{|} \PY{p}{(}\PY{n}{math2018}\PY{o}{.}\PY{n}{studentgroup} \PY{o}{==} \PY{l+s+s1}{\PYZsq{}}\PY{l+s+s1}{AI}\PY{l+s+s1}{\PYZsq{}}\PY{p}{)}\PY{o}{|} \PY{p}{(}\PY{n}{math2018}\PY{o}{.}\PY{n}{studentgroup} \PY{o}{==} \PY{l+s+s1}{\PYZsq{}}\PY{l+s+s1}{AS}\PY{l+s+s1}{\PYZsq{}}\PY{p}{)}\PY{o}{|} \PY{p}{(}\PY{n}{math2018}\PY{o}{.}\PY{n}{studentgroup} \PY{o}{==} \PY{l+s+s1}{\PYZsq{}}\PY{l+s+s1}{FI}\PY{l+s+s1}{\PYZsq{}}\PY{p}{)}\PY{o}{|} \PY{p}{(}\PY{n}{math2018}\PY{o}{.}\PY{n}{studentgroup} \PY{o}{==} \PY{l+s+s1}{\PYZsq{}}\PY{l+s+s1}{HI}\PY{l+s+s1}{\PYZsq{}}\PY{p}{)}\PY{o}{|} \PY{p}{(}\PY{n}{math2018}\PY{o}{.}\PY{n}{studentgroup} \PY{o}{==} \PY{l+s+s1}{\PYZsq{}}\PY{l+s+s1}{PI}\PY{l+s+s1}{\PYZsq{}}\PY{p}{)}\PY{o}{|} \PY{p}{(}\PY{n}{math2018}\PY{o}{.}\PY{n}{studentgroup} \PY{o}{==} \PY{l+s+s1}{\PYZsq{}}\PY{l+s+s1}{WH}\PY{l+s+s1}{\PYZsq{}}\PY{p}{)}\PY{o}{|} \PY{p}{(}\PY{n}{math2018}\PY{o}{.}\PY{n}{studentgroup} \PY{o}{==} \PY{l+s+s1}{\PYZsq{}}\PY{l+s+s1}{MR}\PY{l+s+s1}{\PYZsq{}}\PY{p}{)}\PY{p}{]}\PY{p}{)}
         
         \PY{c+c1}{\PYZsh{}Create a smaller cross table with just the eth/race columns}
         
         \PY{n}{math2018\PYZus{}AvgDistETH}\PY{o}{=}\PY{n}{pd}\PY{o}{.}\PY{n}{crosstab}\PY{p}{(}\PY{n}{math2018ETH}\PY{o}{.}\PY{n}{SchoolType}\PY{p}{,} \PY{n}{math2018ETH}\PY{o}{.}\PY{n}{studentgroup}\PY{p}{,} \PY{n}{values} \PY{o}{=} \PY{n}{math2018ETH}\PY{o}{.}\PY{n}{currstatus}\PY{p}{,} \PY{n}{aggfunc}\PY{o}{=}\PY{l+s+s1}{\PYZsq{}}\PY{l+s+s1}{mean}\PY{l+s+s1}{\PYZsq{}}\PY{p}{)}\PY{o}{.}\PY{n}{round}\PY{p}{(}\PY{l+m+mi}{2}\PY{p}{)}
         \PY{n}{math2018\PYZus{}AvgDistETH}
\end{Verbatim}


\begin{Verbatim}[commandchars=\\\{\}]
{\color{outcolor}Out[{\color{outcolor}73}]:} studentgroup      AA      AI     AS     FI      HI      MR     PI      WH
         SchoolType                                                               
         AltCharter   -169.73 -131.74 -88.80 -30.80 -169.62 -117.75    NaN -131.39
         AltPublic    -220.22     NaN -37.72  37.10 -206.18  -73.73    NaN -159.68
         Charter       -76.66 -102.71  39.17  18.33  -51.83   -7.37 -56.95  -14.02
         Public        -78.07  -81.95  35.99  15.30  -50.76    8.10 -69.44   -9.58
\end{Verbatim}
            
    \begin{Verbatim}[commandchars=\\\{\}]
{\color{incolor}In [{\color{incolor}74}]:} \PY{c+c1}{\PYZsh{}Use it to create bar chart to show smaller set of data (Eth/race)}
         
         \PY{n}{math2018\PYZus{}AvgDistETH}\PY{o}{.}\PY{n}{plot}\PY{p}{(}\PY{n}{kind} \PY{o}{=} \PY{l+s+s1}{\PYZsq{}}\PY{l+s+s1}{bar}\PY{l+s+s1}{\PYZsq{}}\PY{p}{,} \PY{n}{figsize}\PY{o}{=}\PY{p}{(}\PY{l+m+mi}{18}\PY{p}{,}\PY{l+m+mi}{18}\PY{p}{)}\PY{p}{,} \PY{n}{stacked}\PY{o}{=}\PY{k+kc}{False}\PY{p}{)}
\end{Verbatim}


\begin{Verbatim}[commandchars=\\\{\}]
{\color{outcolor}Out[{\color{outcolor}74}]:} <matplotlib.axes.\_subplots.AxesSubplot at 0x2d8067b7fd0>
\end{Verbatim}
            
    \begin{center}
    \adjustimage{max size={0.9\linewidth}{0.9\paperheight}}{output_34_1.png}
    \end{center}
    { \hspace*{\fill} \\}
    
    \begin{Verbatim}[commandchars=\\\{\}]
{\color{incolor}In [{\color{incolor}75}]:} \PY{c+c1}{\PYZsh{}Create an \PYZsq{}programs\PYZsq{} Subgroup to compare like data.}
         
         \PY{n}{math2018PROG}\PY{o}{=}\PY{p}{(}\PY{n}{math2018}\PY{p}{[}\PY{p}{(}\PY{n}{math2018}\PY{o}{.}\PY{n}{studentgroup} \PY{o}{==} \PY{l+s+s1}{\PYZsq{}}\PY{l+s+s1}{EL}\PY{l+s+s1}{\PYZsq{}}\PY{p}{)} \PY{o}{|} \PY{p}{(}\PY{n}{math2018}\PY{o}{.}\PY{n}{studentgroup} \PY{o}{==} \PY{l+s+s1}{\PYZsq{}}\PY{l+s+s1}{ELO}\PY{l+s+s1}{\PYZsq{}}\PY{p}{)}\PY{o}{|} \PY{p}{(}\PY{n}{math2018}\PY{o}{.}\PY{n}{studentgroup} \PY{o}{==} \PY{l+s+s1}{\PYZsq{}}\PY{l+s+s1}{RFP}\PY{l+s+s1}{\PYZsq{}}\PY{p}{)}\PY{o}{|} \PY{p}{(}\PY{n}{math2018}\PY{o}{.}\PY{n}{studentgroup} \PY{o}{==} \PY{l+s+s1}{\PYZsq{}}\PY{l+s+s1}{EO}\PY{l+s+s1}{\PYZsq{}}\PY{p}{)}\PY{o}{|} \PY{p}{(}\PY{n}{math2018}\PY{o}{.}\PY{n}{studentgroup} \PY{o}{==} \PY{l+s+s1}{\PYZsq{}}\PY{l+s+s1}{SED}\PY{l+s+s1}{\PYZsq{}}\PY{p}{)}\PY{o}{|} \PY{p}{(}\PY{n}{math2018}\PY{o}{.}\PY{n}{studentgroup} \PY{o}{==} \PY{l+s+s1}{\PYZsq{}}\PY{l+s+s1}{SWD}\PY{l+s+s1}{\PYZsq{}}\PY{p}{)}\PY{o}{|} \PY{p}{(}\PY{n}{math2018}\PY{o}{.}\PY{n}{studentgroup} \PY{o}{==} \PY{l+s+s1}{\PYZsq{}}\PY{l+s+s1}{FOS}\PY{l+s+s1}{\PYZsq{}}\PY{p}{)}\PY{o}{|} \PY{p}{(}\PY{n}{math2018}\PY{o}{.}\PY{n}{studentgroup} \PY{o}{==} \PY{l+s+s1}{\PYZsq{}}\PY{l+s+s1}{HOM}\PY{l+s+s1}{\PYZsq{}}\PY{p}{)}\PY{p}{]}\PY{p}{)}
         
         \PY{c+c1}{\PYZsh{}Create a smaller cross table with just the Programs columns}
         
         \PY{n}{math2018\PYZus{}AvgDistPROG}\PY{o}{=}\PY{n}{pd}\PY{o}{.}\PY{n}{crosstab}\PY{p}{(}\PY{n}{math2018PROG}\PY{o}{.}\PY{n}{SchoolType}\PY{p}{,} \PY{n}{math2018PROG}\PY{o}{.}\PY{n}{studentgroup}\PY{p}{,} \PY{n}{values} \PY{o}{=} \PY{n}{math2018PROG}\PY{o}{.}\PY{n}{currstatus}\PY{p}{,} \PY{n}{aggfunc}\PY{o}{=}\PY{l+s+s1}{\PYZsq{}}\PY{l+s+s1}{mean}\PY{l+s+s1}{\PYZsq{}}\PY{p}{)}\PY{o}{.}\PY{n}{round}\PY{p}{(}\PY{l+m+mi}{2}\PY{p}{)}
         \PY{n}{math2018\PYZus{}AvgDistPROG}
\end{Verbatim}


\begin{Verbatim}[commandchars=\\\{\}]
{\color{outcolor}Out[{\color{outcolor}75}]:} studentgroup      EL     ELO      EO     FOS     HOM     RFP     SED     SWD
         SchoolType                                                                  
         AltCharter   -185.80 -210.13 -154.00     NaN -189.80 -159.09 -168.76 -201.43
         AltPublic    -222.87 -225.56 -190.54     NaN -213.36 -207.44 -205.23 -196.76
         Charter       -66.37 -103.18  -36.01 -178.90  -75.48  -29.98  -55.79 -107.66
         Public        -57.18  -97.51  -30.02 -123.25  -75.07  -17.77  -51.55 -113.29
\end{Verbatim}
            
    \begin{Verbatim}[commandchars=\\\{\}]
{\color{incolor}In [{\color{incolor}76}]:} \PY{c+c1}{\PYZsh{}use it to create bar chart to show smaller set of data (Eth/race)}
         
         \PY{n}{math2018\PYZus{}AvgDistPROG}\PY{o}{.}\PY{n}{plot}\PY{p}{(}\PY{n}{kind} \PY{o}{=} \PY{l+s+s1}{\PYZsq{}}\PY{l+s+s1}{bar}\PY{l+s+s1}{\PYZsq{}}\PY{p}{,} \PY{n}{figsize}\PY{o}{=}\PY{p}{(}\PY{l+m+mi}{18}\PY{p}{,}\PY{l+m+mi}{18}\PY{p}{)}\PY{p}{,} \PY{n}{stacked}\PY{o}{=}\PY{k+kc}{False}\PY{p}{)}
\end{Verbatim}


\begin{Verbatim}[commandchars=\\\{\}]
{\color{outcolor}Out[{\color{outcolor}76}]:} <matplotlib.axes.\_subplots.AxesSubplot at 0x2d8067fda90>
\end{Verbatim}
            
    \begin{center}
    \adjustimage{max size={0.9\linewidth}{0.9\paperheight}}{output_36_1.png}
    \end{center}
    { \hspace*{\fill} \\}
    

    % Add a bibliography block to the postdoc
    
    
    
    \end{document}
